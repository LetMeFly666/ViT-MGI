% !TEX root = main.tex
\documentclass[conference]{IEEEtran}
\IEEEoverridecommandlockouts
% The preceding line is only needed to identify funding in the first footnote. If that is unneeded, please comment it out.
% \usepackage{fontspec}               % 加载 fontspec 包
\usepackage{xeCJK}                    % 加载 xeCJK 包
% \setCJKmainfont{WenQuanYi Zen Hei}  % 设置中文主字体
\setCJKmainfont[Path=fonts/, Extension=.ttc]{wqy-zenhei}
\usepackage{ulem}      % 删除线
\usepackage{hyperref}  % 超链接
\usepackage{url}       % 处理参考文献中的下划线
% \usepackage{cite}    % 与biblatex包冲突
\usepackage{amsmath,amssymb,amsfonts}
\usepackage{algorithmic}
\usepackage{graphicx}
\usepackage{textcomp}
\usepackage{xcolor}
\def\BibTeX{{\rm B\kern-.05em{\sc i\kern-.025em b}\kern-.08em
    T\kern-.1667em\lower.7ex\hbox{E}\kern-.125emX}}

\usepackage[style=ieee,backend=biber]{biblatex}
\addbibresource{references.bib}
% 替换下划线字符
\DeclareSourcemap{
    \maps[datatype=bibtex]{
        \map{
            \step[
                fieldsource=url,
                match=\regexp{_},
                replace=\regexp{\%5f}
            ]
        }
    }
}

\begin{document}

% 标题
\title{ViT-MGI: Context-aware Lightweight Malicious Gradient Identification for Federated Vision Transformer Systems against Poisoning Attacks\\
{\footnotesize \textsuperscript{*}Note: Sub-titles are not captured in Xplore and
should not be used}
\thanks{Identify applicable funding agency here. If none, delete this.}
}

\author{\IEEEauthorblockN{1\textsuperscript{st} Given Name Surname}
\IEEEauthorblockA{\textit{dept. name of organization (of Aff.)} \\
\textit{name of organization (of Aff.)}\\
City, Country \\
email address or ORCID}
\and
\IEEEauthorblockN{2\textsuperscript{nd} Given Name Surname}
\IEEEauthorblockA{\textit{dept. name of organization (of Aff.)} \\
\textit{name of organization (of Aff.)}\\
City, Country \\
email address or ORCID}
\and
\IEEEauthorblockN{3\textsuperscript{rd} Given Name Surname}
\IEEEauthorblockA{\textit{dept. name of organization (of Aff.)} \\
\textit{name of organization (of Aff.)}\\
City, Country \\
email address or ORCID}
\and
\IEEEauthorblockN{4\textsuperscript{th} Given Name Surname}
\IEEEauthorblockA{\textit{dept. name of organization (of Aff.)} \\
\textit{name of organization (of Aff.)}\\
City, Country \\
email address or ORCID}
\and
\IEEEauthorblockN{5\textsuperscript{th} Given Name Surname}
\IEEEauthorblockA{\textit{dept. name of organization (of Aff.)} \\
\textit{name of organization (of Aff.)}\\
City, Country \\
email address or ORCID}
\and
\IEEEauthorblockN{6\textsuperscript{th} Given Name Surname}
\IEEEauthorblockA{\textit{dept. name of organization (of Aff.)} \\
\textit{name of organization (of Aff.)}\\
City, Country \\
email address or ORCID}
}

\maketitle

% 摘要
\begin{abstract}

% 联邦学习可以在数据不离开客户端的前提下进行模型训练,保护了用户的隐私且减小了中央服务器的压力。然而,在联邦学习的过程中,经常会有恶意客户端的存在。当然,恶意用户的数量一般会占据较小的部分。本文对联邦学习过程中用户上传上来的梯度变化进行分析,首先利用最大池化技术提取主要特征的同时降低,利用主成分分析(PCA)算法\textcolor{red}{(这里还有待添加更多的算法)}对恶意用户进行识别。单次被识别为恶意用户可能是由于误判,因此本文使用主观逻辑模型来对用户的信用等级进行评估,并依据用户的可信度来调整模型聚合时的权重。

% 结果表明,当前方法对于恶意用户的识别率、恶意用户的识别效率,以及程序自身的鲁棒性都有所提升。我们将其命名为FLDefinder\textcolor{red}{(名字待定)}并发布了其源代码,以方便该领域的未来研究:\href{https://github.com/LetMeFly666/FLDefinder}{https://github.com/LetMeFly666/FLDefinder}\textcolor{red}{(论文投稿前此仓库为Private状态不可访问)}。

随着视觉大模型的不断发展,模型训练时需要越来越多的数据量。因此需要联邦学习的ViT模型(Federated ViT),在数据不离开多个客户端的前提下进行模型训练,同时捕捉复杂的全局特征\footnote{相比于简单的CNN等,ViT可以更好地捕捉全局信息}。例如FeSTA通过分割学习的ViT模型进行COVID-19的胸部X光片检测,保留数据隐私的同时在多个数据集上实现了性能提升\cite{federatedViT_example}。但是,在实际的应用场景中,可能会存在恶意攻击者攻击污染全局模型的问题。有的攻击者会篡改本地训练出的梯度数据,从而扰乱聚合后的全局模型的效果;有的攻击者会篡改数据集的标签,例如常见的标签翻转\cite{tailAttack_SuchAsLabelFlip};还有的攻击者会采用更加隐蔽的后门攻击,在训练过程中涉及难以被识别的触发器从而达到插入后门的效果\cite{backdoor_001}。

现已提出了很多防御机制,有通过在服务器上计算每个模型更新与其最近更新之间的欧氏距离之和来选择聚合模型的Krum算法和拓展的multi-Krum算法\cite{aggregation_Krum},有选择中值或排除边缘值后的平均值作为全局模型的中值算法和裁剪平均算法\cite{aggregation_MedianTrimmedMean},也有使用主成分分析(Principal Component Analysis, PCA)的Fed-PCA算法\cite{Fed_PCA},以及通过解决最大团问题(Maximum CliqueProblem, MCP)从而无需无需恶意用户数量这一先验知识的Sniper方案\cite{aggregation_Sniper}。这些数据都能在不同程度上解决恶意用户的攻击问题,但是它们都普遍存在安全和效率冲突的问题。即,安全性较高的算法计算复杂度普遍较高,而计算效率高的算法其安全性未必能得到很好的保证。Fedteratred ViT下这些方法存在效率低、不鲁棒等缺陷。很小的ViT-B/16模型也有86百万级别的参数,相比于参数级别为百万甚至十万的传统CNN模型,即使是线性复杂度的安全检测方法,恶意攻击的检测耗时也要提升几十倍甚至几百倍\footnote{这一句在说参数量大所以原有防御方案效率低}。在庞大的ViT模型的参数中有很多不是那么敏感的成分,这些部分恶意用户于正常用户的区别一般不大,从而可能会降低恶意检测的鲁棒性\footnote{这一句在说参数散多所以原有防御方案鲁棒性差}。

In this paper,我们提出了一种针对ViT的两阶段的上下文感知轻量级恶意梯度识别方法来提升恶意梯度检测时的效率和鲁棒性。通过放大池化技术\footnote{最终也有可能会决定使用PCA来分析出主成分}和特征层提取技术,在减少计算量的同时去除无效的梯度信息,从而在提升了检测效率的同时提升了恶意检测的鲁棒性。Specifically,对于用户上传上来的梯度信息,首先进行PCA算法对数据进行降维,将数据量下降到原来的4\%\footnote{这个值是目前最佳的实验结果,后续可能会变,记得回来修改},去除无用信息并保留有效信息,同时为后续算法降低了需要处理的数据量大小。随后我们使用隔离森林算法随机挑选剩下梯度中的1000\footnote{具体数值同样待定}个特征,进一步降低恶意识别部分的复杂度。由于已经使用过主成分分析算法,因此隔离森林算法的输入数据特征都比较明显,因此可以纯随机挑选要计算的维度。最后,我们使用主观逻辑模型\cite{Subjective_Logic_Model}对用户评分并在聚合的过程中附以权重,这样使得聚合结果更加有效的同时减少了由于随机导致的错误封禁\footnote{因为多次都错误封禁一个恶意用户的概率会指数级别的减小}。这样,在ViT这种具有大量参数的模型下,也能够进行很好的联邦学习训练并杜绝可能的潜在的攻击。

我们在CIFAR-10、CIFAR-100、xxx\footnote{待确认}上做了有关模型效率以及有关识别鲁棒性的实验,在处理时间上相比先进的xx\footnote{待确认,找个最近提出的但复杂度较高的算法}算法降低了80\%。模型鲁棒性方面,单纯的隔离森林的识别准确率非常低,这是由梯度中有大量的接近0的值所导致的。而对于使用放大池化来减少计算量并增强鲁棒性的算法\cite{betterTogether},虽然能极大程度地减少计算量,但在ViT的拜占庭攻击下对鲁棒性会起到相反的效果。相比于单纯的PCA算法,我们的识别准确率由70\%提升到了95\%。

我们将其命名为ViT-MGI并发布了其源代码,以方便该领域的未来研究:\href{https://github.com/LetMeFly666/FLDefinder}{https://github.com/LetMeFly666/FLDefinder}\footnote{论文投稿前此仓库为Private状态不可访问}。

% 下面是关于摘要的大纲

% Abstract:

% 随着视觉大模型的发展,训练需要更多数据量,所以要做Fedteratred ViT。(可举例)

% 但是这里面会存在恶意攻击者污染xx(待扩展,比如xx攻击 实际例子也可)。

% (问题)现有很多防御也提出了xx(不用太多)。However存在(安全效率冲突)问题。(可解释:)更复杂、数据更离散...   导致了现有防御方案在Fedteratred ViT下针对它们的防御存在效率低(xx模型参数多少 耗时要提高xx倍 )下,不鲁棒xxx缺陷(可举例) 。(比如一下xxxx万参数,我xx的池化限制到xx,(和CNN差不多))

% (方法) In this paper,我们提出了一种两阶段的xx方法来提升效率、鲁棒性,通过xx理念解决xx问题(一句话:核心思想)。具体来说Specifically,首先通过xxx来xx了xx\%,随后我们xxx,最终/从而 实现了xx(具体是xx方法,别虚话)。

% (效果)我们在多少数据集上做了多少实验,从xx方面显示我们比xx等(当前最新)防御方法取得了xx提升。(若实验很好可多写点:在效率上提升了xx倍,在精度上xx  (详述))

\end{abstract}


\begin{IEEEkeywords}

Federated Learning, Gradient Analysis, Malicious User Detection, Subjective Logic Model\footnote{关键词待定}
\end{IEEEkeywords}

\section{Introduction}

\label{sec:intro}

% 联邦学习(FL)使参与者能够联合训练一个全局模型,而不需要共享他们的数据集\cite{BetterTogether24, BetterTogether29}。

% \textcolor{red}{上一段其实是抄的BetterTogether}

如今,边缘设备的激增使得数据量爆炸式增长\cite{edgeComputing_explosiveGrowth},数据隐私的重要性也正显得越来越重要。2024年,全球大约75\%的人口将受到现代隐私法规的保护。这些法规的扩展推动了隐私实践的实施,并且企业在数据隐私方面的年均预算预计将超过250万美元\cite{dataPrivacyIsEncreasing}。为了解决这一问题,一种越来越流行的方式是使用联邦学习\cite{useFL2solve}。联邦学习(Federated Learning, FL)是一种分布式机器学习方法,旨在在多个分散的数据源上进行协作训练,同时避免将数据集中存储。FL由McMahan等人于2016年提出,通过在客户端设备上进行本地模型训练,并仅上传模型更新到中央服务器进行聚合,从而保护数据隐私和安全。一旦聚合,服务器将更新后的模型广播回所有客户端。节点将在本地更新其模型,并使用以下更新与新一批本地数据。这种方法不仅减少了数据传输的需求,还能在不暴露原始数据的情况下,利用分散的数据源提高模型性能\cite{FLGenesisArticle}\footnote{这段Intro参考Mitigating Adversarial写的}。

联邦学习中一个流行的训练模型是Transformer\cite{transformer},Transformer模型引入了自注意力机制(self-attention mechanism),能够在处理输入序列时捕捉长距离依赖关系,从而显著提高了模型的性能和并行处理能力,并且能够有效地被运用在翻译\cite{transformer_translation}、音频分类\cite{transformer_audioClassification}、计算机视觉(ViT)\cite{transformer_vision}等多个领域。联邦学习中对ViT进行训练或微调引起了业界和学术界的广泛兴趣\cite{transformer_gotInterest},该系列模型需要大量的设计以及强大的计算资源,因此在训练过程中需要特别注意保护模型的完整性。\footnote{这段Intro参考Mitigating Adversarial写的}

假设存在一些恶意客户端,在客户端向服务器发送训练得到的梯度之前,首先对梯度进行更改,例如对梯度取反,从而严重破环全局模型的训练质量。或者会存在一些恶意客户端,篡改本地数据,诱导中央服务器学到一些错误特征从而达到植入后门的效果\footnote{这里要看最终是否实现了backdoor的攻防}。这些行为都会严重影响最终的训练结果,甚至会造成一些安全性的问题。

In this work,我们专注于对针对恶意用户上传恶意梯度从而攻击服务器模型进行系统的防御。图1\footnote{TODO: 画一张小图}描述了在正常训练中存在梯度上升攻击的恶意用户的情况。图的左侧是参与训练的客户端,它们首先接收中央服务器广播下发的全局模型,然后使用本地数据对模型进行一定量的训练,之后再将训练得到的梯度变化汇总到中央服务器上,从而完成一轮总的训练。假设在参与训练的客户端中出现了一些恶意用户,它们出于一定的目的想要干扰中央服务器的模型训练结果。于是它们在将训练得到的梯度变化上传到中央服务器之前,先对这些梯度进行一些修改操作,例如将梯度取反等,从而试图下降总模型的训练效果。

In this Paper,我们提出了ViT-MGI:对于如图所示的恶意用户的攻击行为,ViT-MGT是一种有效的防御手段。它首先对模型上传上来的梯度进行主成分萃取操作,减少后续识别所需数据量的同时,成功剔除了大量无效的信息。这些“无效”信息是由于ViT模型的体积较大所导致的,对于单次训练,大量(例如ViT-B/16具有86百万级别的参数量)的梯度变化中,有很多梯度的变化几乎为0。如果不进行主成分萃取操作,这些对于攻击识别几乎无效的大量数据不仅会增加后续计算的复杂度,还会增加攻击者和正常参与者梯度之间的相似程度。在进行主成分萃取以及后续的识别时,我们只关注ViT模型的patch\_embed层、attn层以及mlp层\footnote{待定},因为我们通过实验I\footnote{记得最后写好实验后增加引用链接}得知,存在梯度上升、标签翻转、后门植入等常见的攻击时,这些层的变化最为敏感,而其他层变化普遍较小,甚至有几乎全0的层变化的存在。经过一系列的处理,剩下的就只有我们所关注的敏感的少量的梯度变化了。最后,使用异常检测手段例如隔离森林,就能轻松准确地识别出恶意攻击的存在。考虑到实际训练过程中可能会存在很多随机因素,因此可能会出现一定的错误识别现象,即一些正常用户的梯度偶尔可能会和恶意用户较为相似,从而导致此轮次被错误识别为恶意行为用户,我们使用主观逻辑模型来对每个用户评分,依据用户权重来聚合模型的同时,也减少了由于随机因素导致的错误封禁。正常用户单次被识别为恶意的概率很小,多次连续被识别为恶意的概率更是会呈指数级别地减小。

Our main contributions are as follows:

\begin{enumerate}
    \item 我们表明,通过数据降维,我们可以在极高效率的前提下完成恶意梯度的识别。
    \item 我们使用ViT-MGI进行关键数据提取,从而使得恶意梯度的识别能达到较高的精度。
    \item 我们的ViT-MGI能够结合多轮训练结果跟随并感知恶意用户的行为,从而提升了整体的鲁棒性。
    \item 我们设计了一系列巧妙的实验证明了ViT-MGI的有效性。
\end{enumerate}

The rest of the paper is as follows. \hyperref[sec:related]{§\ref{sec:related}}surveys related work. \hyperref[sec:model]{§\ref{sec:model}} presents our system model. \hyperref[sec:method]{§\ref{sec:method}} describes the principles of the ViT-MGI shielding defense. We evaluate it on ViT-B/16 against a gradient-based attack and discuss the results in \hyperref[sec:exp]{§\ref{sec:exp}}. Finally, we conclude in \hyperref[sec:conclusion]{§\ref{sec:conclusion}}.


\vspace{0.25cm}
\rule{\linewidth}{0.4pt}
\centerline{\Large\bf{以下为编写大纲}}
\vspace{1cm}

Introduction:

(创新点:核心-效率、精度-跟随-感知、实验)

列几点(至少上面的3点):

\begin{enumerate}
    \item ssf
    \item sfs
\end{enumerate}

一段话的本文结论 第二章是xx,section 3是xxx。(可参考IEEE transction)





Related work:

(找十几篇,每个一两句话)

(先分两到三类  每类一段 )

第一段xx攻击(可一句话带过)

第二段 针对这种攻击有做xx的有做xx的 (总起)之后细分 大概分几类(2-3)

一段一类。有头有尾(开头 这种方法基于xx,   一句一篇(也可加一句好坏之类的),   结尾 总结这类方法也存在xx问题(可从被引的文章里找来改写))

最后一小段几句话(可写可不写)总体来说  (扣一下主题)   通过上述分析,核心问题是xx(就是摘要里的),因此我们要xx来解决。


\textcolor{red}{今晚写好,(conclusion)}


System model:

(建模) 

一部分是FL ViT场景 (一张图,至少包含←这个,可包含攻击)  数学定义之类的   可参考别人描述

(也可放参数表,\textcolor{red}{一般不放})

一部分是攻击模型   

(描述清楚,逻辑清楚就好,没有前面那么地受约束)

\textcolor{red}{今晚至少构思好}




Methodology:

(也是总分,技术上是怎么实现的)

(至少一张图,可2张3张,一张大的(总的方案图),可加小(细节)图)

(最后也可给个伪代码 )

\textcolor{red}{明后天写完}


Experiments:

实验设置(用的xx机器,在xx跑,用的xx模型xx参数,对比的哪些攻击/防御,数据集,...)能一大段

实验结果分析(分点)

比如:
1. 精度
2. 速度
3. 防御效果
对比着给点结论。

\textcolor{red}{一直跑着实验}
(最好写的部分)


Conclusion:

摘要 + 实验结果


\section{Related Work}
\label{sec:related}

相关工作

\section{System Model}
\label{sec:model}

系统模型

\section{Methodology}
\label{sec:method}

方法

\section{Experiments}
\label{sec:exp}

实验

\section{Conclusion}
\label{sec:conclusion}

结论。


\section{Ease of Use}

\subsection{Maintaining the Integrity of the Specifications}

The IEEEtran class file is used to format your paper and style the text. All margins, 
column widths, line spaces, and text fonts are prescribed; please do not 
alter them. You may note peculiarities. For example, the head margin
measures proportionately more than is customary. This measurement 
and others are deliberate, using specifications that anticipate your paper 
as one part of the entire proceedings, and not as an independent document. 
Please do not revise any of the current designations.

\section{Prepare Your Paper Before Styling}
Before you begin to format your paper, first write and save the content as a 
separate text file. Complete all content and organizational editing before 
formatting. Please note sections \ref{AA}--\ref{SCM} below for more information on 
proofreading, spelling and grammar.

Keep your text and graphic files separate until after the text has been 
formatted and styled. Do not number text heads---{\LaTeX} will do that 
for you.

\subsection{Abbreviations and Acronyms}\label{AA}
Define abbreviations and acronyms the first time they are used in the text, 
even after they have been defined in the abstract. Abbreviations such as 
IEEE, SI, MKS, CGS, ac, dc, and rms do not have to be defined. Do not use 
abbreviations in the title or heads unless they are unavoidable.

\subsection{Units}
\begin{itemize}
\item Use either SI (MKS) or CGS as primary units. (SI units are encouraged.) English units may be used as secondary units (in parentheses). An exception would be the use of English units as identifiers in trade, such as ``3.5-inch disk drive''.
\item Avoid combining SI and CGS units, such as current in amperes and magnetic field in oersteds. This often leads to confusion because equations do not balance dimensionally. If you must use mixed units, clearly state the units for each quantity that you use in an equation.
\item Do not mix complete spellings and abbreviations of units: ``Wb/m\textsuperscript{2}'' or ``webers per square meter'', not ``webers/m\textsuperscript{2}''. Spell out units when they appear in text: ``. . . a few henries'', not ``. . . a few H''.
\item Use a zero before decimal points: ``0.25'', not ``.25''. Use ``cm\textsuperscript{3}'', not ``cc''.)
\end{itemize}

\subsection{Equations}
Number equations consecutively. To make your 
equations more compact, you may use the solidus (~/~), the exp function, or 
appropriate exponents. Italicize Roman symbols for quantities and variables, 
but not Greek symbols. Use a long dash rather than a hyphen for a minus 
sign. Punctuate equations with commas or periods when they are part of a 
sentence, as in:
\begin{equation}
a+b=\gamma\label{eq}
\end{equation}

Be sure that the 
symbols in your equation have been defined before or immediately following 
the equation. Use ``\eqref{eq}'', not ``Eq.~\eqref{eq}'' or ``equation \eqref{eq}'', except at 
the beginning of a sentence: ``Equation \eqref{eq} is . . .''

\subsection{\LaTeX-Specific Advice}

Please use ``soft'' (e.g., \verb|\eqref{Eq}|) cross references instead
of ``hard'' references (e.g., \verb|(1)|). That will make it possible
to combine sections, add equations, or change the order of figures or
citations without having to go through the file line by line.

Please don't use the \verb|{eqnarray}| equation environment. Use
\verb|{align}| or \verb|{IEEEeqnarray}| instead. The \verb|{eqnarray}|
environment leaves unsightly spaces around relation symbols.

Please note that the \verb|{subequations}| environment in {\LaTeX}
will increment the main equation counter even when there are no
equation numbers displayed. If you forget that, you might write an
article in which the equation numbers skip from (17) to (20), causing
the copy editors to wonder if you've discovered a new method of
counting.

{\BibTeX} does not work by magic. It doesn't get the bibliographic
data from thin air but from .bib files. If you use {\BibTeX} to produce a
bibliography you must send the .bib files. 

{\LaTeX} can't read your mind. If you assign the same label to a
subsubsection and a table, you might find that Table I has been cross
referenced as Table IV-B3. 

{\LaTeX} does not have precognitive abilities. If you put a
\verb|\label| command before the command that updates the counter it's
supposed to be using, the label will pick up the last counter to be
cross referenced instead. In particular, a \verb|\label| command
should not go before the caption of a figure or a table.

Do not use \verb|\nonumber| inside the \verb|{array}| environment. It
will not stop equation numbers inside \verb|{array}| (there won't be
any anyway) and it might stop a wanted equation number in the
surrounding equation.

\subsection{Some Common Mistakes}\label{SCM}
\begin{itemize}
\item The word ``data'' is plural, not singular.
\item The subscript for the permeability of vacuum $\mu_{0}$, and other common scientific constants, is zero with subscript formatting, not a lowercase letter ``o''.
\item In American English, commas, semicolons, periods, question and exclamation marks are located within quotation marks only when a complete thought or name is cited, such as a title or full quotation. When quotation marks are used, instead of a bold or italic typeface, to highlight a word or phrase, punctuation should appear outside of the quotation marks. A parenthetical phrase or statement at the end of a sentence is punctuated outside of the closing parenthesis (like this). (A parenthetical sentence is punctuated within the parentheses.)
\item A graph within a graph is an ``inset'', not an ``insert''. The word alternatively is preferred to the word ``alternately'' (unless you really mean something that alternates).
\item Do not use the word ``essentially'' to mean ``approximately'' or ``effectively''.
\item In your paper title, if the words ``that uses'' can accurately replace the word ``using'', capitalize the ``u''; if not, keep using lower-cased.
\item Be aware of the different meanings of the homophones ``affect'' and ``effect'', ``complement'' and ``compliment'', ``discreet'' and ``discrete'', ``principal'' and ``principle''.
\item Do not confuse ``imply'' and ``infer''.
\item The prefix ``non'' is not a word; it should be joined to the word it modifies, usually without a hyphen.
\item There is no period after the ``et'' in the Latin abbreviation ``et al.''.
\item The abbreviation ``i.e.'' means ``that is'', and the abbreviation ``e.g.'' means ``for example''.
\end{itemize}
An excellent style manual for science writers is \cite{b7}.

\subsection{Authors and Affiliations}
\textbf{The class file is designed for, but not limited to, six authors.} A 
minimum of one author is required for all conference articles. Author names 
should be listed starting from left to right and then moving down to the 
next line. This is the author sequence that will be used in future citations 
and by indexing services. Names should not be listed in columns nor group by 
affiliation. Please keep your affiliations as succinct as possible (for 
example, do not differentiate among departments of the same organization).

\subsection{Identify the Headings}
Headings, or heads, are organizational devices that guide the reader through 
your paper. There are two types: component heads and text heads.

Component heads identify the different components of your paper and are not 
topically subordinate to each other. Examples include Acknowledgments and 
References and, for these, the correct style to use is ``Heading 5''. Use 
``figure caption'' for your Figure captions, and ``table head'' for your 
table title. Run-in heads, such as ``Abstract'', will require you to apply a 
style (in this case, italic) in addition to the style provided by the drop 
down menu to differentiate the head from the text.

Text heads organize the topics on a relational, hierarchical basis. For 
example, the paper title is the primary text head because all subsequent 
material relates and elaborates on this one topic. If there are two or more 
sub-topics, the next level head (uppercase Roman numerals) should be used 
and, conversely, if there are not at least two sub-topics, then no subheads 
should be introduced.

\subsection{Figures and Tables}
\paragraph{Positioning Figures and Tables} Place figures and tables at the top and 
bottom of columns. Avoid placing them in the middle of columns. Large 
figures and tables may span across both columns. Figure captions should be 
below the figures; table heads should appear above the tables. Insert 
figures and tables after they are cited in the text. Use the abbreviation 
``Fig.~\ref{fig}'', even at the beginning of a sentence.

\begin{table}[htbp]
\caption{Table Type Styles}
\begin{center}
\begin{tabular}{|c|c|c|c|}
\hline
\textbf{Table}&\multicolumn{3}{|c|}{\textbf{Table Column Head}} \\
\cline{2-4} 
\textbf{Head} & \textbf{\textit{Table column subhead}}& \textbf{\textit{Subhead}}& \textbf{\textit{Subhead}} \\
\hline
copy& More table copy$^{\mathrm{a}}$& &  \\
\hline
\multicolumn{4}{l}{$^{\mathrm{a}}$Sample of a Table footnote.}
\end{tabular}
\label{tab1}
\end{center}
\end{table}

% \begin{figure}[htbp]
% \centerline{\includegraphics{fig1.png}}
% \caption{Example of a figure caption.}
% \label{fig}
% \end{figure}

Figure Labels: Use 8 point Times New Roman for Figure labels. Use words 
rather than symbols or abbreviations when writing Figure axis labels to 
avoid confusing the reader. As an example, write the quantity 
``Magnetization'', or ``Magnetization, M'', not just ``M''. If including 
units in the label, present them within parentheses. Do not label axes only 
with units. In the example, write ``Magnetization (A/m)'' or ``Magnetization 
\{A[m(1)]\}'', not just ``A/m''. Do not label axes with a ratio of 
quantities and units. For example, write ``Temperature (K)'', not 
``Temperature/K''.

\section*{Acknowledgment}

The preferred spelling of the word ``acknowledgment'' in America is without 
an ``e'' after the ``g''. Avoid the stilted expression ``one of us (R. B. 
G.) thanks $\ldots$''. Instead, try ``R. B. G. thanks$\ldots$''. Put sponsor 
acknowledgments in the unnumbered footnote on the first page.

\section*{References}

Please number citations consecutively within brackets \cite{maxwell1892}. The 
sentence punctuation follows the bracket \cite{b2}. Refer simply to the reference 
number, as in \cite{b3}---do not use ``Ref. \cite{b3}'' or ``reference \cite{b3}'' except at 
the beginning of a sentence: ``Reference \cite{b3} was the first $\ldots$''

Number footnotes separately in superscripts. Place the actual footnote at 
the bottom of the column in which it was cited. Do not put footnotes in the 
abstract or reference list. Use letters for table footnotes.

Unless there are six authors or more give all authors' names; do not use 
``et al.''. Papers that have not been published, even if they have been 
submitted for publication, should be cited as ``unpublished'' \cite{b4}. Papers 
that have been accepted for publication should be cited as ``in press'' \cite{b5}. 
Capitalize only the first word in a paper title, except for proper nouns and 
element symbols.

For papers published in translation journals, please give the English 
citation first, followed by the original foreign-language citation \cite{b6}.

% \begin{thebibliography}{00}  % 花括号中的 00 决定了在文献编号中所需要的宽度
% \bibitem{b1} G. Eason, B. Noble, and I. N. Sneddon, ``On certain integrals of Lipschitz-Hankel type involving products of Bessel functions,'' Phil. Trans. Roy. Soc. London, vol. A247, pp. 529--551, April 1955.
% \bibitem{b2} J. Clerk Maxwell, A Treatise on Electricity and Magnetism, 3rd ed., vol. 2. Oxford: Clarendon, 1892, pp.68--73.
% \bibitem{b3} I. S. Jacobs and C. P. Bean, ``Fine particles, thin films and exchange anisotropy,'' in Magnetism, vol. III, G. T. Rado and H. Suhl, Eds. New York: Academic, 1963, pp. 271--350.
% \bibitem{b4} K. Elissa, ``Title of paper if known,'' unpublished.
% \bibitem{b5} R. Nicole, ``Title of paper with only first word capitalized,'' J. Name Stand. Abbrev., in press.
% \bibitem{b6} Y. Yorozu, M. Hirano, K. Oka, and Y. Tagawa, ``Electron spectroscopy studies on magneto-optical media and plastic substrate interface,'' IEEE Transl. J. Magn. Japan, vol. 2, pp. 740--741, August 1987 [Digests 9th Annual Conf. Magnetics Japan, p. 301, 1982].
% \bibitem{b7} M. Young, The Technical Writer's Handbook. Mill Valley, CA: University Science, 1989.
% \end{thebibliography}
\printbibliography % 打印参考文献
\vspace{12pt}
\color{red}
IEEE conference templates contain guidance text for composing and formatting conference papers. Please ensure that all template text is removed from your conference paper prior to submission to the conference. Failure to remove the template text from your paper may result in your paper not being published.

\end{document}
